% vim: set textwidth=120:

% Example CV based on the 1.5-column-cv template. Main features:
% * uses the Roboto font family and IcoMoon icon set;
% * doesn't use colours, different font weights are used instead for styling;
% * because the CV fits on one page, header and footer is empty, since there isn't much useful info to put there;
% * includes a photo.
\documentclass[a4paper,10pt]{article}


% package imports
% ---------------

\usepackage[british]{babel} % for correct language and hyphenation and stuff
\usepackage{calc}           % for easier length calculations (infix notation)
\usepackage{enumitem}       % for configuring list environments
\usepackage{fancyhdr}       % for setting header and footer
\usepackage{fontspec}       % for fonts
\usepackage{geometry}       % for setting margins (\newgeometry)
\usepackage{graphicx}       % for pictures
\usepackage{microtype}      % for microtypography stuff
\usepackage[dvipsnames]{xcolor}         % for colours
\usepackage{hyperref}


% margin and column widths
% ------------------------

% margins
\newgeometry{left=15mm,right=15mm,top=15mm,bottom=15mm}

% width of the gap between left and right column
\newlength{\cvcolumngapwidth}
\setlength{\cvcolumngapwidth}{3.5mm}

% left column width
\newlength{\cvleftcolumnwidth}
\setlength{\cvleftcolumnwidth}{45mm}

% right column width
\newlength{\cvrightcolumnwidth}
\setlength{\cvrightcolumnwidth}{\textwidth-\cvleftcolumnwidth-\cvcolumngapwidth}

% set paragraph indentation to 0, because it screws up the whole layout otherwise
\setlength{\parindent}{0mm}


% style definitions
% -----------------
% style categories explanation:
% * \cvnameXXX is used for the name;
% * \cvsectionXXX is used for section names (left column, accompanied by a horizontal rule);
% * \cvtitleXXX is used for job/education titles (right column);
% * \cvdurationXXX is used for job/education durations (left column);
% * \cvheadingXXX is used for headings (left column);
% * \cvmainXXX (and \setmainfont) is used for main text;
% * \cvruleXXX is used for the horizontal rules denoting sections.

% font families
\defaultfontfeatures{Ligatures=TeX} % reportedly a good idea, see https://tex.stackexchange.com/a/37251

\newfontfamily{\cvnamefont}{Roboto Medium}
\newfontfamily{\cvsectionfont}{Roboto Medium}
\newfontfamily{\cvtitlefont}{Roboto Regular}
\newfontfamily{\cvdurationfont}{Roboto Light Italic}
\newfontfamily{\cvheadingfont}{Roboto Regular}
\setmainfont{Roboto Light}

% colours
\definecolor{cvnamecolor}{HTML}{000000}
\definecolor{cvsectioncolor}{rgb}{0.0, 0.48, 0.65}
\definecolor{cvtitlecolor}{HTML}{000000}
\definecolor{cvdurationcolor}{HTML}{000000}
\definecolor{cvheadingcolor}{HTML}{000000}
\definecolor{cvmaincolor}{HTML}{000000}
% \definecolor{cvrulecolor}{HTML}{000000}
\definecolor{cvrulecolor}{rgb}{0.0, 0.48, 0.65}
%\definecolor{blue(ncs)}{rgb}{0.0, 0.53, 0.74}
% \definecolor{cerulean}{rgb}{0.0, 0.48, 0.65}

\color{cvmaincolor}

% styles
\newcommand{\cvnamestyle}[1]{{\Large\cvnamefont\textcolor{cvnamecolor}{#1}}}
\newcommand{\cvsectionstyle}[1]{{\normalsize\cvsectionfont\textcolor{cvsectioncolor}{#1}}}
\newcommand{\cvtitlestyle}[1]{{\large\cvtitlefont\textcolor{cvtitlecolor}{#1}}}
\newcommand{\cvdurationstyle}[1]{{\small\cvdurationfont\textcolor{cvdurationcolor}{#1}}}
\newcommand{\cvheadingstyle}[1]{{\normalsize\cvheadingfont\textcolor{cvheadingcolor}{#1}}}


% inter-item spacing
% ------------------

% vertical space after personal info and standard CV items
\newlength{\cvafteritemskipamount}
\setlength{\cvafteritemskipamount}{5mm plus 1.25mm minus 1.25mm}

% vertical space after sections
\newlength{\cvaftersectionskipamount}
\setlength{\cvaftersectionskipamount}{2mm plus 0.5mm minus 0.5mm}

% extra vertical space to be used when a section starts with an item with a heading (e.g. in the skills section),
% so that the heading does not follow the section name too closely
\newlength{\cvbetweensectionandheadingextraskipamount}
\setlength{\cvbetweensectionandheadingextraskipamount}{1mm plus 0.25mm minus 0.25mm}


% intra-item spacing
% ------------------

% vertical space after name
\newlength{\cvafternameskipamount}
\setlength{\cvafternameskipamount}{3mm plus 0.75mm minus 0.75mm}

% vertical space after personal info lines
\newlength{\cvafterpersonalinfolineskipamount}
\setlength{\cvafterpersonalinfolineskipamount}{2mm plus 0.5mm minus 0.5mm}

% vertical space after titles
\newlength{\cvaftertitleskipamount}
\setlength{\cvaftertitleskipamount}{1mm plus 0.25mm minus 0.25mm}

% value to be used as parskip in right column of CV items and itemsep in lists (same for both, for consistency)
\newlength{\cvparskip}
\setlength{\cvparskip}{0.5mm plus 0.125mm minus 0.125mm}

% set global list configuration (use parskip as itemsep, and no separation otherwise)
\setlist{parsep=0mm,topsep=0mm,partopsep=0mm,itemsep=\cvparskip}


% CV commands
% -----------

% creates a "personal info" CV item with the given left and right column contents, with appropriate vertical space after
% @param #1 left column content (should be the CV photo)
% @param #2 right column content (should be the name and personal info)
\newcommand{\cvpersonalinfo}[2]{
     %left and right column
    \begin{minipage}[t]{\cvleftcolumnwidth}
       \vspace{0mm} % XXX hack to align to top, see https://tex.stackexchange.com/a/11632
       \raggedleft #1
    \end{minipage}% XXX necessary comment to avoid unwanted space
    \hspace{\cvcolumngapwidth}% XXX necessary comment to avoid unwanted space
    \begin{minipage}[t]{\cvrightcolumnwidth}
        \vspace{0mm} % XXX hack to align to top, see https://tex.stackexchange.com/a/11632
        #2
    \end{minipage}

    % space after
    \vspace{\cvafteritemskipamount}
}

% typesets a name, with appropriate vertical space after
% @param #1 name text
\newcommand{\cvname}[1]{
    % name
    \cvnamestyle{#1}

    % space after
    \vspace{\cvafternameskipamount}
}

% typesets a line of personal info beginning with an icon, with appropriate vertical space after
% @param #1 parameters for the \includegraphics command used to include the icon
% @param #2 icon filename
% @param #3 line text
\newcommand{\cvpersonalinfolinewithicon}[3]{
    % text
    #3
    % icon, vertically aligned with text (see https://tex.stackexchange.com/a/129463)
    \raisebox{.5\fontcharht\font`E-.5\height}{\includegraphics[#1]{#2}}
    

    % space after
    \vspace{\cvafterpersonalinfolineskipamount}
}

% creates a "section" CV item with the given left column content, a horizontal rule in the right column, and with
% appropriate vertical space after
% @param #1 left column content (should be the section name)
\newcommand{\cvsection}[1]{
    % left and right column
    \begin{minipage}[t]{\cvleftcolumnwidth}
        \raggedleft\cvsectionstyle{#1}
    \end{minipage}% XXX necessary comment to avoid unwanted space
    \hspace{\cvcolumngapwidth}% XXX necessary comment to avoid unwanted space
    \begin{minipage}[t]{\cvrightcolumnwidth}
        \textcolor{cvrulecolor}{\rule{\cvrightcolumnwidth}{0.6mm}}
    \end{minipage}

    % space after
    \vspace{\cvaftersectionskipamount}
}

% creates a standard, multi-purpose CV item with the given left and right column contents, parskip set to cvparskip
% in the right column, and with appropriate vertical space after
% @param #1 left column content
% @param #2 right column content
\newcommand{\cvitem}[2]{
    % left and right column
    \begin{minipage}[t]{\cvleftcolumnwidth}
        \raggedleft #1
    \end{minipage}% XXX necessary comment to avoid unwanted space
    \hspace{\cvcolumngapwidth}% XXX necessary comment to avoid unwanted space
    \begin{minipage}[t]{\cvrightcolumnwidth}
        \setlength{\parskip}{\cvparskip} #2
    \end{minipage}

    % space after
    \vspace{\cvafteritemskipamount}
}

% typesets a title, with appropriate vertical space after
% @param #1 title text
\newcommand{\cvtitle}[1]{
    % title
    \cvtitlestyle{#1}

    % space after
    \vspace{\cvaftertitleskipamount}
    % XXX need to subtract cvparskip here, because it is automatically inserted after the title "paragraph"
    \vspace{-\cvparskip}
}

\newcommand{\emptyline}{\vspace{1mm}}


% header and footer
% -----------------

% set empty header and footer
\pagestyle{empty}



% preamble end/document start
% ===========================

\begin{document}


% personal info
% -------------

\cvpersonalinfo{
    % photo
    \includegraphics[height=10mm]{blank.png}
}{
    % name
    \cvname{\hspace*{\fill}Matias San Martin Acosta}

    % address
    %\cvpersonalinfolinewithicon{height=4mm}{072-location.pdf}{
    %    Mendoza 2178. Piñeyro, Avellaneda. CP1870
    %}

    % phone number
    \hspace*{\fill}\cvpersonalinfolinewithicon{height=4mm}{067-phone.pdf}{
    +54\,9\,11\,4436\,9212
    }

    % email address
    \hspace*{\fill}\cvpersonalinfolinewithicon{height=4mm}{070-envelop.pdf}{
        \href{mailto:msanmartinacosta@gmail.com}{msanmartinacosta@gmail.com}
    }

    % LinkedIn account
    \hspace*{\fill}\cvpersonalinfolinewithicon{height=4mm}{458-linkedin.pdf}{
       \href{https://www.linkedin.com/in/msanmartinacosta}{msanmartinacosta}
    }

    % date of birth
    % Born 15 February 1989
}


% work experience
% ---------------

\cvsection{EXPERIENCIA LABORAL}

% Redbee Studios
\cvitem{
    \cvdurationstyle{Agosto 2017 -- presente}
}{
    \cvtitle{Backend Developer -- \normalsize Redbee Studios, para Prisma Medios de Pago} \emptyline

    \begin{itemize}[leftmargin=*]
    
        \item Análisis, diseño y desarrollo para la arquitectura de microservicios de TodoPago. Utilizando tecnologías como Java 1.8, Spring Boot, Redis, Apache Kafka, Couchbase, SQLServer, Git, Gradle, Maven, Groovy, Spock Framework \emptyline
        
        \item Desarrollo utilizando metodologías ágiles, particularmente SCRUM. Participación y moderación de plannings, refinamiento de backlog, code reviews y retrospectivas. Se utilizan también herramientas como pair-programming y mob-programming \emptyline
        
    \end{itemize}
}

% Capgemini
\cvitem{
    \cvdurationstyle{Julio 2016 -- Agosto 2017}
}{
    \cvtitle{Fullstack Developer -- \normalsize Capgemini, para PSA Peugeot Citroen \emptyline}

    \begin{itemize}[leftmargin=*]
    
        \item Portail VI -- Java 1.8, JPA, AngularJS, Bootstrap, Tomcat 8, MySQL DB, Spring Batch, JUnit. DDD with Seedstackand w20. Diseño, desarrollo, análisis funcional, estimaciones \emptyline
        
        \item DFCM -- Java 1.7, Hibernate, Struts 2, Spring MVC, Spring Security, JavaScript, jQuery, dhtmlx, Apache Tiles,jQuery UI, jUnit-DbUnit, Oracle DB, Tomcat 8, Maven 3. Diseño, desarrollo, análisis funcional, estimaciones \emptyline
        
        \item FLUCOMEX -- Java 1.5, Apache OJB, Struts 1, Struts-Layout, JavaScript, jQuery, Oracle DB (PL/SQL), Tomcat 5.5. Desarrollo y testing unitario\emptyline
        
        \item CISA -- Java 1.7, JPA, Web Services REST, Oracle DB (PL/SQL). Desarrollo y testing unitario
        
    \end{itemize}
}

% ICBC
\cvitem{
    \cvdurationstyle{Febrero 2015 -- Julio 2016}
}{
    \cvtitle{Fullstack Developer -- \normalsize Certant S.A, para ICBC - Industrial and Commercial Bank of China \emptyline}
    
    \begin{itemize}[leftmargin=*]
        \item Multipay -- Java 1.4, HTML, Struts, JSP, JavaScript, jQuery, Base de Datos Oracle SQL, WebSphere ApplicationServer 5. Desarrollo, análisis, resolución de incidentes, diseño de tests y análisis de cobertura, capación e inducción a trainee's \emptyline
        
        \item AAMW -- Java 1.7, jUnit, Hibernate, HTML, Spring, Spring MVC, Spring Security, Javascript, jQuery, WebSphere Application Server 8.5. Desarrollo y test unitarios
        
    \end{itemize}
}

% education
% ---------

\cvsection{EDUCACIÓN}

% FIUBA
\cvitem{
    \cvdurationstyle{2007 -- presente}
}{
    \cvtitle{Ingeniería Informática -- \normalsize Facultad de Ingeniería, Universidad de Buenos Aires}
     
}

%\newpage

%Cursos
\cvsection{CURSOS}

\cvitem{
    \cvdurationstyle{}
}{
    \cvtitle{Udemy \normalsize \href{https://www.udemy.com/user/matias-san-martin-acosta/}{https://www.udemy.com/user/matias-san-martin-acosta/}}
    
    \begin{itemize}[leftmargin=*]
    
        \item \emph{Completados}
            \begin{itemize}[leftmargin=*]
                \item \href{https://www.udemy.com/certificate/UC-VANR3KSW/}{Aprende Docker de cero a experto}
                \item \href{https://www.udemy.com/certificate/UC-Y0TQHFO5/}{Java Multithreading, Concurrency and Performance Optimization}
            \end{itemize}
        \item \emph{En progreso}
            \begin{itemize}[leftmargin=*]
            \item \href{https://www.udemy.com/angular-2-fernando-herrera/}{Angular 7+}
            \item \href{https://www.udemy.com/course/rock-the-jvm-scala-for-beginners/}{Rock the JVM! Scala and Functional Programming for beginners}
            \end{itemize}
        
    \end{itemize}
}

% Idiomas
\cvsection{IDIOMAS}

\cvitem{
    \cvdurationstyle{}
}{
    \cvtitle{Inglés}
    \begin{itemize}[leftmargin=*]
        \item Lectura, escritura, comprensión oral: Advanced, Conversación: Upper-Intermediate
    \end{itemize}
}

\end{document}